\section{Auswertung}
\label{sec:Auswertung}

\subsection{Statische Methode}

Aus den Abbildung (1) und (2) im Anhang, die die Graphen der Temperaturzunahme der jeweils weiter vom Heizelement entfernten Messstellen $T_\text{1}$ und $T_\text{4}$ und $T_\text{5}$ und $T_\text{8}$ zeigen, lässt sich als gemeinsames Merkmal ein zunächst exponentieller und dann immer weiter abflachender Anstieg ablesen, der sich einem bestimmten Temperaturwert annähert.
Die Verläufe von $T_\text{1}$ und $T_\text{4}$ sind sich sehr ähnlich, während $T_\text{5}$ und $T_\text{8}$ stark von einander abweichen.\newline
Dies lässt sich auch an den nach 700 Sekunden gemessenen Werten
\begin{equation*}
T_\text{1}=316.46\SI{\kelvin}\newline
T_\text{4}=314.43\SI{\kelvin}\newline
T_\text{5}=320.15\SI{\kelvin}\newline
T_\text{8}=305.88\SI{\kelvin}
\end{equation*}
erkennen.
Thermoelement $T_\text{5}$ des Aluminiumstabes misst dabei die höchste Temperatur.
Zu Beginn der Messreihe hatten alle Proben in etwa dieselbe Temperatur, was darauf schließen lässt, das Aluminium die höchste Wärmeleitfähigkeit $\kappa$ der getesteten Stoffe besitzt.
Der Wärmestrom pro Zeit $\frac{\mathrm{d}Q}{\mathrm{d}t}$
lässt sich nach Formel\eqref{eq:} berechnen. Die gemessenen Werte lassen sich gemeinsam mit den für $\frac{\mathrm{d}Q}{\mathrm{d}t}$ berechneten Werten in Tabelle \ref{tab:tab1} ablesen. Die Graphen der Temperaturdifferenzen $T_\text{2}-T_\text{1}$ und $T_\text{7}-T_\text{8}$ sind in Abbildung (3) und (4) im Anhang zu sehen.
Es werden die Wärmeleitungsfähigkeit $\kappa$ aus der Literatur\cite{kappa} und die Querschnitsfläche $A$ aus der Anleitung\cite{V204} entnommen. Dabei wird für $\kappa_\text{Messing}$ der Mittelwert des angegeben Bereichs genommen.
Die Distanz zwischen den Thermoelementen eines Stabes $dx$ wurde gemessen:
\[
dx = (0.03\pm0.00)\SI{\metre}\newline
\kappa_\text{Messing} = 93\SI{\Watt\per\meter\per\kelvin}\newline
\kappa_\text{Edelstahl} = 20\SI{\Watt\per\meter\per\kelvin}\newline
A_\text{Messing, breit} = A_\text{Edelstahl} = 48*10^{-6}\SI{\metre^{2}}
\]

\begin{table}
	\centering
	\caption{Die gemessenen Daten für Temperaturdifferenzen und den Wärmestrom pro Zeit zum Zeitpunkt $t$.}
		\sisetup{table-format=1.2}
	\begin{tabular}{S[table-format=3.2] S[table-format=3.2]S[table-format=3.2]S[table-format=3.2]}
		\toprule
		{$t/\si[per-mode=reciprocal]{\second}$}&{$T_\text{2-1}/\si[per-mode=reciprocal]{\kelvin}$}&{$\frac{\mathrm{d}Q_\text{2-1}}{\mathrm{d}t}/\si[per-mode=reciprocal]{\Watt}$}&{$T_\text{7-8}/\si[per-mode=reciprocal]{\kelvin}$}&{$\frac{\mathrm{d}Q_\text{7-8}}{\mathrm{d}t}/\si[per-mode=reciprocal]{\Watt}$}\\
		\midrule
		5 & 0.359 & -5.34 & -0.082 & 0.26 \\
		200 & 4.49 & -66.81 & 11.9 & -38.08 \\
		400 & 3.07 & -45.68 & 10.3 & -32.96 \\
		550 & 2.81 & -41.81 & 9.65 & -30.88 \\
		950 & 2.69 & -40.03 & 8.91 & -28.51 \\
		\bottomrule
	\end{tabular}
	\label{tab:tab1}
\end{table}
Beide Graphen steigen zunächst zu einem Maximalwert an und fallen anschließend wieder, wobei sie sich wieder einem Temperaturwert annähern.
Jedoch liegt das Maximum des Stahlstabes (s.Anhang, Abb.4)
deutlich höher, als das des Messingstabes (s.Anhang, Abb.3).
Des Weiteren ist der Abfall nach erreichen des Maximums bei beiden Graphen in absoluten Zahlen zwar in etwa gleich.
Dies hängt mit den unterschiedlichen Wärmeleitfähigkeiten $\kappa$ zusammen:
Da Messing eine wesentlich höhere Wärmeleitfähigkeit hat, als Edelstahl verteilt sich die Wärme im Messingstab schneller und die Temperaturdifferenz wird schneller ausgeglichen.
Das Maximum entsteht dadurch, dass zu Beginn die Heizung stärker ist als die Wärmeleitfähigkeit.