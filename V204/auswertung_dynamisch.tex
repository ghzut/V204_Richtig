\subsection{Dynamische Methode}
Für die dynamischen Methode wird $\kappa$ über Formel \eqref{eq:}
berechnet die gemessenen Werte für die Periodendauer $\mathrm{d}t$ und für die beiden Amplituden $A_\text{nah}$ und $A_\text{fern}$ des näheren und des weiter entfernteren Thermoelements lassen sich aus Tabelle \ref{tab:tab2}, \ref{tab:tab3} und \ref{tab:tab4}ablesen. Die Materialkonstanten $\rho$und $c$ werden aus der Anleitung entnommen\cite{V204}. Die Distanz $dx$ ist dabei dieselbe, wie bei der statischen Methode.
Der Fehler für $\kappa$ errechnet sich dabei über Formel \eqref{eq:}
\begin{table}
	\centering
	\caption{Temperatur des breiten Messingstabs mit Periodendauer 80 s.}
	\sisetup{table-format=1.2}
	\begin{tabular}{S[table-format=3.2] S[table-format=3.2]S[table-format=3.2]S[table-format=3.2]}
		\toprule
		{$\mathrm{d}t/\si[per-mode=reciprocal]{\second}$}&{$A_\text{nah}/\si[per-mode=reciprocal]{\kelvin}$}&{$A_\text{fern}/\si[per-mode=reciprocal]{\kelvin}$} \\
		\midrule
		13.33 & 7.27 & 3.41 \\
		11.11 & 6.36 & 3.18 \\
		13.33 & 6.14 & 2.73 \\
		13.33 & 5.91 & 2.50 \\
		13.33 & 5.68 & 2.27 \\
		13.33 & 5.45 & 2.27 \\
		13.33 & 5.45 & 2.05 \\
		17.78 & 5.45 & 1.82 \\
		17.78 & 5.23 & 1.82 \\
		17.78 & 5.23 & 1.82 \\
		\bottomrule
	\end{tabular}
	\label{tab:tab2}
\end{table}
Für Messing ergibt sich damit \[\kappa = \SI{(119.60\pm27.97)}{\Watt\per\metre\per\kelvin}\].
\begin{table}
	\centering
	\caption{Temperatur des breiten Aluminiumstabs mit Periodendauer 80 s.}
	\sisetup{table-format=1.2}
	\begin{tabular}{S[table-format=3.2] S[table-format=3.2]S[table-format=3.2]S[table-format=3.2]}
		\toprule
		{$\mathrm{d}t/\si[per-mode=reciprocal]{\second}$}&{$A_\text{nah}/\si[per-mode=reciprocal]{\kelvin}$}&{$A_\text{fern}/\si[per-mode=reciprocal]{\kelvin}$} \\
		\midrule
		8.89 & 10.20 & 6.60 \\
		8.89 & 8.80 & 5.40 \\
		8.89 & 8.20 & 4.80 \\
		6.67 & 8.00 & 4.40 \\
		11.11 & 7.80 & 4.00 \\
		8.89 & 7.60 & 4.00 \\
		6.67 & 7.60 & 3.80 \\
		8.89 & 7.40 & 3.80 \\
		8.89 & 7.40 & 3.60 \\
		8.89 & 7.20 & 3.60 \\
		\bottomrule
	\end{tabular}
	\label{tab:tab3}
\end{table}
Für Aluminium ergibt sich damit \[\kappa = \SI{(205.33\pm40.67)}{\Watt\per\metre\per\kelvin}\].
\begin{table}
	\centering
	\caption{Temperatur des breiten Edelstahlstabs mit Periodendauer 200 s.}
	\sisetup{table-format=1.2}
	\begin{tabular}{S[table-format=3.2] S[table-format=3.2]S[table-format=3.2]S[table-format=3.2]}
		\toprule
		{$\mathrm{d}t/\si[per-mode=reciprocal]{\second}$}&{$A_\text{nah}/\si[per-mode=reciprocal]{\kelvin}$}&{$A_\text{fern}/\si[per-mode=reciprocal]{\kelvin}$} \\
		\midrule
		30.30 & 13.33 & 3.56 \\
		36.36 & 11.33 & 3.11 \\
		30.30 & 10.67 & 2.67 \\
		36.36 & 10.22 & 2.22 \\
		36.36 & 10.00 & 2.00 \\
		\bottomrule
	\end{tabular}
	\label{tab:tab4}
\end{table}
Für Edelstahl ergibt sich damit \[\kappa = \SI{(30.28\pm4.45)}{\Watt\per\metre\per\kelvin}\].