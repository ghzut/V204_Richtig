
\section{Diskussion}
\label{sec:Diskussion}

\subsection{Statische Methode}

Aus den Abbildung (1) und (2) im Anhang, die die Graphen der Temperaturzunahme der jeweils weiter vom Heizelement entfernten Messstellen $T_\text{1}$ und $T_\text{4}$ und $T_\text{5}$ und $T_\text{8}$ zeigen, lässt sich als gemeinsames Merkmal ein zunächst exponentieller und dann immer weiter abflachender Anstieg ablesen, der sich einem bestimmten Temperaturwert annähert.
Die Verläufe von $T_\text{1}$ und $T_\text{4}$ sind sich sehr ähnlich, während $T_\text{5}$ und $T_\text{8}$ stark von einander abweichen.\newline
Dies lässt sich auch an den nach $\SI{700}{\second}$ gemessenen Werten erkennen.
Thermoelement $T_\text{5}$ des Aluminiumstabes misst dabei die höchste Temperatur.
Zu Beginn der Messreihe haben alle Proben in etwa dieselbe Temperatur, was darauf schließen lässt, das Aluminium die höchste Wärmeleitfähigkeit $\kappa$ der getesteten Stoffe besitzt.

Die Graphen aus Abbildung (3) und (4) steigen zunächst zu einem Maximalwert an und fallen anschließend wieder, wobei sie sich wieder einem Temperaturwert annähern.
Jedoch liegt das Maximum des Stahlstabes (s.Anhang, Abb.4)
deutlich höher, als das des Messingstabes (s.Anhang, Abb.3).
Des Weiteren ist der Abfall nach erreichen des Maximums bei beiden Graphen in absoluten Zahlen zwar in etwa gleich.
Dies hängt mit den unterschiedlichen Wärmeleitfähigkeiten $\kappa$ zusammen:
Da Messing eine wesentlich höhere Wärmeleitfähigkeit hat, als Edelstahl verteilt sich die Wärme im Messingstab schneller und die Temperaturdifferenz wird schneller ausgeglichen.
Das Maximum entsteht dadurch, dass zu Beginn die Heizung stärker ist als die Wärmeleitfähigkeit.

\subsection{Dynamische Methode}

Die über die dynamische Methode berechneten Mittelwerte für $\kappa$ weichen von den Literaturwerten, die in der statischen Methode verwendet werden ab. Die relativen Fehler betragen
\[\Delta\kappa_.{Messing}=29,03%\newline
\Delta\kappa_.{Aluminium}=6,81%\newline
\Delta\kappa_.[Edelstahl}=50%.
\]
Auffällig sind auch die großen Standardabweichungen der Mittelwerte.\newline
Beides lässt sich darauf zurückführen, dass sämtliche Werte der Tabellen \ref{tab:tab1}, \ref{tab:tab2} und \ref{tab:tab3} von Hand in den Graphen der Abbildungen (5), (6) und (7) gemessen wurden.